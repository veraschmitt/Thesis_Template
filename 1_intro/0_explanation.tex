\subsubsection{main.tex}
Here you can change the overall structure of the chapters. If you go to the main file from line 104 on the content of the chapters is defined. With the input command you can insert the sections from the different folders. The headings of the sections and subsections should be added/defined in the main file not in the respective sub-files for the different chapters. Here also the list of figures and tables are added from line 86 on. if you add a figure or table in your text it should appear in the list automatically. 

\subsubsection{Chapter}
The different chapters are organized in folders which makes a longer writing project better structured. Just write the text into the different .tex files which you can add and remove via the main.tex file. 

\subsubsection{Figures}
In the figures folder you can add all your figures in order to keep the structure more concise. .png and .pdf are always working, some other formats might cause some errors. 

\subsubsection{References}
All your references should be added to the \textbf{ref.bib} file. There are some examples of how to add different formats of citations for inproceedings or books and so forth. If you are not familiar with the citation styles a good hint is, if you look for the publication on Google Scholar and then click on the "cite" button which are the two tick marks. There you can see different citation styles and at the bottom there is the BibTex button. Click on it and you will be redirected to the the latex citation format. Check if everything is correct (sometimes there are some errors) and then just copy that and add it to the ref.bib file. For this document the APA citation style is selected, as it makes it more understandable for the reader. If you want to change it, you can change line 63 in the main.tex file to another format.

\subsubsection{Rest Folder}
Here you can find the \textbf{declaration.tex} file which appears in the end of the thesis and should be signed by you. This is crucial and mandatory for any plagiary related issues. Hereby you state that the thesis is your own work. You can simply add your details in the file directly and if you print out the thesis you can either sign it manually or also with any kind or program you are usually working with.\\

The \textbf{glossary.tex} file defines all your abbreviations, this is sometimes useful, when you use a lot of abbreviations to collect them in one place, such that the reader can quickly look them up, if he/she has forgotten what the abbreviation means.

You can define all your abbreviations in the \textbf{glossary.tex} file and if you use the abbreviation for the first time in the text than use following command: \gls{gru}), this will automatically add the abbreviation in the visible "List of Abbreviations". 
E.g. currently there are many abbreviations in the glossary.tex file but only 2 (ME and GRU) are used in the text. only if you use the gls command the abbreviation will be shown in the printed file. \\

The \textbf{title.tex} file defines everything on the title page. Here you can add your name and the names of your supervisors, the title and subtitle of your thesis an so forth. The logo of the TU is already added and you actually just need to fill in your data. The date is always the actual date of the day you are working in the document. \\

The file \textbf{list.tex} is an example of a very long table and how it best should look like. Please use the predefined style of the tables, as this is the most prominent one in scientific writing. \\

The \textbf{notes.tex} file is for you personally, to take some general notes.\\


\textbf{You can delete the 0.explanation.tex} file in the intro folder when you do not need it anymore, or you can just simply remove it from the main.tex file. Then it will not show up in your text anymore and the introduction starts with numbering 1. Furthermore you can remove the list.tex file from the main file if you do not need such a long table as an example and you can out-commend or remove all the explanations and examples given. Furthermore, I kept all the references in the ref.bib file as they are used in the list.tex file you can just simply delete all the bib items there or create another bib file.
